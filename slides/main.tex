\documentclass{beamer}
\usetheme{Montpellier}
\usecolortheme{beaver}
\usepackage{verbatim}	
\usepackage[square,sort]{natbib}
\usepackage{comment}
\usepackage{amsmath}
\usepackage{caption}
\usepackage{hyperref}

\setbeamerfont{footnote}{size=\tiny}
\setbeamerfont{footnote mark}{size=\tiny}
\setbeamerfont{caption}{size=\scriptsize}
\setbeamerfont{cite}{size=\tiny}


\title{Team 4 : Population Dynamics}

\author{Brady Butler\inst{1} \& Sajid Ali\inst{2}}	
\institute 
{\inst{1}	
	Physics\\
	University of Maine\\
	\inst{2}%
	Applied Physics\\
	Northwestern University}
\date{\today}

\pgfdeclareimage[height=0.5cm]{university-logo}{pearc19}
\logo{\pgfuseimage{university-logo}}

\AtBeginSubsection[]
{
  \begin{frame}<beamer>{Outline}
    \tableofcontents[currentsection,currentsubsection]
  \end{frame}
}

% Let's get started
\begin{document}


\begin{frame}
  \titlepage
\end{frame}

\begin{frame}{Outline}
  \tableofcontents
  % You might wish to add the option [pausesections]
\end{frame}

% Section and subsections will appear in the presentation overview
% and table of contents.

\section{Lokta-Voletra!}
\begin{frame}{Predator-Prey ``Relationship" \footnote{www.alaskapublic.org,en.wikipedia.org}}
\begin{block}{}
	\begin{columns}[onlytextwidth,T]
		\column{\dimexpr\linewidth-30mm-10mm}
		\begin{itemize}
		\begin{figure}
			\vspace*{-1.1cm}\hspace*{-1.1cm}\includegraphics[width=50mm]{../plot_notebooks/moose_apr}
		\end{figure}
		\end{itemize}
		\column{30mm}
		\begin{figure}
			\vspace*{1.1cm}\hspace*{-1.1cm}\includegraphics[width=50mm]{../plot_notebooks/wiki_wolf}
		\end{figure}
	\end{columns}
\end{block}
\end{frame}

\section{Results}
\begin{frame}{Predator-Prey Model}
	\item Coupled ODE's describing behavior of predator\& prey populations.
	\item
$\frac{dM}{dt}=br_{M}*M-df_{M}*M*W \ |\  \frac{dW}{dt}=br_{W}*W*M-df_{W}*W$	
		\begin{figure}
		\includegraphics[scale=0.35]{../plot_notebooks/short_and_long_runs}
		\caption{Stability analysis}
		\end{figure}
\end{frame}

\begin{frame}{Sensitivity to Moose Birth and Death rates}
	\item `Amplitude' plots for Moose and Wolf populations.\footnote{Square root of quantities visualized for enhanced contrast}
	\begin{figure}
		\vspace*{-0.25cm}\hspace*{-1.1cm}\includegraphics[scale=0.35]{../plot_notebooks/sqrt_amp_vary_M}
	\end{figure}
\end{frame}

\begin{frame}{Sensitivity to Moose Birth and Death rates}
	\item `Period' plots for Moose and Wolf populations.\footnote{Square root of quantities visualized for enhanced contrast}
	\begin{figure}
		\vspace*{-0.25cm}\hspace*{-1.1cm}\includegraphics[scale=0.35]{../plot_notebooks/sqrt_per_vary_M}
	\end{figure}
\end{frame}


\begin{frame}{Sensitivity to Wolf Birth and Death rates}
	\item `Amplitude' plots for Moose and Wolf populations.\footnote{Square root of quantities visualized for enhanced contrast}
	\begin{figure}
		\vspace*{-0.25cm}\hspace*{-1.1cm}\includegraphics[scale=0.35]{../plot_notebooks/sqrt_amp_vary_W}
	\end{figure}
\end{frame}

\begin{frame}{Sensitivity to Wolf Birth and Death rates}
	\item `Period' plots for Moose and Wolf populations.\footnote{Square root of quantities visualized for enhanced contrast}
	\begin{figure}
		\vspace*{-0.25cm}\hspace*{-1.1cm}\includegraphics[scale=0.35]{../plot_notebooks/sqrt_per_vary_W}
	\end{figure}\end{frame}


\section{Real-World Factors}
\begin{frame}{Model Carrying Capacity}
	\begin{itemize}
		\item What is carrying capacity ?
		\item Which plot ?
	\end{itemize}
\end{frame}


\begin{frame}{Explaining real world data}
	\begin{itemize}
		\item We can fit our model to real world data
		\item Which plot ?
	\end{itemize}
\end{frame}

\section{Implementation}
\begin{frame}{Language \& Libraries!}
	\begin{itemize}
		\item Libraries : 
		   \begin{itemize}
			\item Numpy/Scipy : Arrays and Fit functions
			\item Matplotlib : Visualization
			\item Pandas  : Dataframes
			\end{itemize}
		\item How to HPC ?
		   \begin{itemize}
			\item Numexpr : Fast expression evaluation, to optimize code
			\item Joblib : Parallel execution module
			\item HDF5 : Structured I/O for scalable performance.
			\item More science ! ?
			\end{itemize}
	\end{itemize}
\end{frame}


\section{Conclusion}
\begin{frame}{Conclusion}
	\begin{itemize}
		\item And thus an important project was undertaken.
		\item Thank you to the organizers of PEARC19! 
	\end{itemize}
\end{frame}


% All of the following is optional and typically not needed. 
%\appendix
%\renewcommand*{\bibfont}{\small}
%\section{References}
%\begin{frame}[t, allowframebreaks]
%\frametitle{References}
%\bibliographystyle{alpha}
%\bibliography{fd}
%\end{frame}

\end{document}

